%%%%%%%%%%%%%%%%%%%%%%%%%%%%%%%%%%%%%%%
% Deedy - One Page Two Column Resume
% LaTeX Template
% Version 1.2 (16/9/2014)
%
% Original author:
% Debarghya Das (http://debarghyadas.com)
%
% Original repository:
% https://github.com/deedydas/Deedy-Resume
%
% IMPORTANT: THIS TEMPLATE NEEDS TO BE COMPILED WITH XeLaTeX
%
% This template uses several fonts not included with Windows/Linux by
% default. If you get compilation errors saying a font is missing, find the line
% on which the font is used and either change it to a font included with your
% operating system or comment the line out to use the default font.
% 
%%%%%%%%%%%%%%%%%%%%%%%%%%%%%%%%%%%%%%
% 
% TODO:
% 1. Integrate biber/bibtex for article citation under publications.
% 2. Figure out a smoother way for the document to flow onto the next page.
% 3. Add styling information for a "Projects/Hacks" section.
% 4. Add location/address information
% 5. Merge OpenFont and MacFonts as a single sty with options.
% 
%%%%%%%%%%%%%%%%%%%%%%%%%%%%%%%%%%%%%%
%
% CHANGELOG:
% v1.1:
% 1. Fixed several compilation bugs with \renewcommand
% 2. Got Open-source fonts (Windows/Linux support)
% 3. Added Last Updated
% 4. Move Title styling into .sty
% 5. Commented .sty file.
%
%%%%%%%%%%%%%%%%%%%%%%%%%%%%%%%%%%%%%%%
%
% Known Issues:
% 1. Overflows onto second page if any column's contents are more than the
% vertical limit
% 2. Hacky space on the first bullet point on the second column.
%
%%%%%%%%%%%%%%%%%%%%%%%%%%%%%%%%%%%%%%


\documentclass[]{deedy-resume-openfont}
\usepackage{fancyhdr}
 
\pagestyle{fancy}
\fancyhf{}
 
\begin{document}

%%%%%%%%%%%%%%%%%%%%%%%%%%%%%%%%%%%%%%
%
%     LAST UPDATED DATE
%
%%%%%%%%%%%%%%%%%%%%%%%%%%%%%%%%%%%%%%
\lastupdated

%%%%%%%%%%%%%%%%%%%%%%%%%%%%%%%%%%%%%%
%
%     TITLE NAME
%
%%%%%%%%%%%%%%%%%%%%%%%%%%%%%%%%%%%%%%
\namesection{}{Emilio Molina}

%%%%%%%%%%%%%%%%%%%%%%%%%%%%%%%%%%%%%%
%
%     COLUMN ONE
%
%%%%%%%%%%%%%%%%%%%%%%%%%%%%%%%%%%%%%%

\begin{minipage}[t]{0.30\textwidth} 

%%%%%%%%%%%%%%%%%%%%%%%%%%%%%%%%%%%%%%
%     EDUCATION
%%%%%%%%%%%%%%%%%%%%%%%%%%%%%%%%%%%%%%

%%%%%%%%%%%%%%%%%%%%%%%%%%%%%%%%%%%%%%
%     LINKS
%%%%%%%%%%%%%%%%%%%%%%%%%%%%%%%%%%%%%%

\section{Contact} 
Phone: {\bf +34 610373998}\\
Mail: \href{mailto:emilio.mol.mar@gmail.com}{\bf emilio.mol.mar@gmail.com}\\
Web: \urlstyle{same}\href{https://emilio-molina.github.io/}{\bf https://emilio-molina.github.io}\\
Github: \href{https://github.com/emilio-molina}{\bf /emilio-molina} \\
LinkedIn: \href{https://www.linkedin.com/in/emilio-molina-martinez/}{\bf /emilio-molina-martinez}


%%%%%%%%%%%%%%%%%%%%%%%%%%%%%%%%%%%%%%
%     COURSEWORK
%%%%%%%%%%%%%%%%%%%%%%%%%%%%%%%%%%%%%%


\vspace{0.5cm}
\section{Courses \& others}
\href{https://confirm.udacity.com/ZW9LLAGZ}{Deep Learning Nanodegree Foundation (Udacity, 2018)} \\
Professional Degree of Classical Piano\\(Conservatori del Liceu, Barcelona, 2011)
\vspace{0.5cm}
%%%%%%%%%%%%%%%%%%%%%%%%%%%%%%%%%%%%%%
%     SKILLS
%%%%%%%%%%%%%%%%%%%%%%%%%%%%%%%%%%%%%%

\section{Techs}
Unix \textbullet{}   Git \textbullet{} Docker \textbullet{} AWS \textbullet{} GCP \textbullet{} MySQL \textbullet{} MongoDB \textbullet{} Jenkins \textbullet{} Bash
\textbullet{} Python \textbullet{} Django \textbullet{} Flask \textbullet{} FastAPI \textbullet{} uWSGI
\textbullet{} Pandas \textbullet{} Numpy \textbullet{} Keras \textbullet{} TensorFlow
\textbullet{} Scikit-learn \textbullet{} Jupyter \textbullet{} Matplotlib
\textbullet{} Bokeh \textbullet{} Librosa \textbullet{} Essentia
\textbullet{} OpenCV
\textbullet{} Pytest \textbullet{} Unittest \textbullet{} Flake8
\textbullet{} C++ \textbullet{} Boost \textbullet{} FFTW3 \textbullet{} TensorFlow C++ API \textbullet{} pybind11 \textbullet{} VTune \textbullet{} Cpplint \textbullet{} HTML \textbullet{} Bootstrap \textbullet{} CSS \textbullet{} Javascript \textbullet{} jQuery \textbullet{} Java \textbullet{} Lucene \textbullet{} Tomcat

\vspace{0.5cm}
\section{Languages}
{\bf Spanish}: mother tongue \\
{\bf English}: Certificate in Advanced English (CAE) (University of Cambridge, July 2013) \\
{\bf French}: DELF B2 French Certificate (May 2010)
(Universit´e de Lille) \\

\sectionsep

%%%%%%%%%%%%%%%%%%%%%%%%%%%%%%%%%%%%%%
%
%     COLUMN TWO
%
%%%%%%%%%%%%%%%%%%%%%%%%%%%%%%%%%%%%%%

\end{minipage} 
\hfill
\begin{minipage}[t]{0.66\textwidth} 

%%%%%%%%%%%%%%%%%%%%%%%%%%%%%%%%%%%%%%
%     EXPERIENCE
%%%%%%%%%%%%%%%%%%%%%%%%%%%%%%%%%%%%%%

\section{Experience}
\runsubsection{BMAT}
\descript{| Research Engineer }
\location{Jan 2015 - Present | Barcelona / Remote}
\vspace{\topsep} % Hacky fix for awkward extra vertical space
\begin{tightemize}
\item Leading role in a research \& development team of 6 members.
\item Research and development on new audio fingerprinting approaches for background music (+20\% id-rate), and for pitch-shifted music. In production.
\item Porting of legacy Java matching algorithm to C++11 / Python bindings / HTTP API in Flask. Runtime optimization (4x gain) in collaboration of HPC research team from UAB. In production.
\item Development of system for record linkage and entity resolution of noisy music metadata from multiple sources. Implemented in Python with HTTP API (FastApi). In production.
\item Research and development of music detection system using machine learning. Develop web annotation tool in Django, prepared training dataset, and manage annotation of it with external freelances. Best algorithm in \href{https://www.music-ir.org/mirex/wiki/2018:Music_and_or_Speech_Detection_Results}{MIREX 2018}. Implemented in C++11 with TensorFlow C++ API. In production.
\item Technical management of several publicly-funded research projects
\item Prepared multiple demos and PoCs of new tech ideas for potential customers
\end{tightemize}
\sectionsep

\runsubsection{Universitat Pompeu Fabra}
\descript{| Assistant Professor }
\location{October 2020 – December 2021 | Barcelona / Remote}
\begin{tightemize}
\item Part-time assistant professor for subject ``Audio and Video Encoding Systems''
\end{tightemize}
\sectionsep

\runsubsection{Universidad de Málaga}
\descript{| Research staff }
\location{January 2012 – January 2015 | Málaga}
\begin{tightemize}
\item Research and development (mostly in Matlab) of algorithms on music information retrieval and audio processing
\item Publication of multiple journal and conference papers, leading to a PhD thesis. List in \href{https://emilio-molina.github.io/publications}{https://emilio-molina.github.io/publications}
\end{tightemize}
\sectionsep

%%%%%%%%%%%%%%%%%%%%%%%%%%%%%%%%%%%%%%
%     RESEARCH
%%%%%%%%%%%%%%%%%%%%%%%%%%%%%%%%%%%%%%


%%%%%%%%%%%%%%%%%%%%%%%%%%%%%%%%%%%%%%
%     AWARDS
%%%%%%%%%%%%%%%%%%%%%%%%%%%%%%%%%%%%%%

\section{Education}
\descript{PhD, Music Information Retrieval }
\location{2012 - 2017 | Universidad de Málaga}
%\vspace{\topsep} % Hacky fix for awkward extra vertical space
\begin{tightemize}
\item E.Molina, "Singing Information Processing: Techniques and Applications", PhD Thesis. University of Málaga, 2017.
\end{tightemize}
\descript{Master's Degree, Sound and Music Computing}
\location{2010 - 2012 | Universitat Pompeu Fabra, Barcelona}
%\vspace{\topsep} % Hacky fix for awkward extra vertical space
\begin{tightemize}
\item Audio signal analysis and processing. Machine learning. Time series analysis. Music Production. 
\end{tightemize}
\descript{Telecommunication Engineering}
\location{2007 - 2012 | Universidad de Málaga}
%\vspace{\topsep} % Hacky fix for awkward extra vertical space
\begin{tightemize}
\item Dissertation
``Software tool for the correction of dissonances in polyphonic music'', awarded
the Finalist Ericsson National Prize on Applications for Multimedia Environments in May 2013
\item Erasmus studentship at Polytech Lille (France) (Sept. 2009 - June 2011).
\end{tightemize}
\descript{Technical Degree in Telecommunications, Sound and Image}
\location{2003 - 2007 | Universidad de Málaga}
%\vspace{\topsep} % Hacky fix for awkward extra vertical space
\begin{tightemize}
\item Dissertation
``Multitrack environment for audio recording in C\#'', awarded the Best Project in Technical Telecommunication Engineering at University of Málaga
\item Award for Best Academic Record
\end{tightemize}

\sectionsep

\sectionsep


\end{minipage} 
\end{document}  \documentclass[]{article}

